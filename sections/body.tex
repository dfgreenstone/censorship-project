\section{Body}\label{sec:body}

To determine whether Internet censorship still exists in Taiwan and in what form, I looked at data from the global measurement platforms OONI, ICLab,
and Censored Planet. This strategy was suggested in the paper "GFWeb: Measuring the Great Firewall's Web Censorship at Scale" by Nguyen Phong Hoang et al.,
although the authors of said paper did note that each of those platforms has its own blind spots, and so this analysis should be understood to be highly
preliminary. (The authors surmounted those blind spots by building their own system to test China's Great Firewall, but I have neither the time nor the
technical knowledge to do the same here.)

As such, we will go through the tests recorded on each of these measurement platforms in Taiwan, beginning with OONI.

\subsection{OONI: Internet Connectivity Tests}
Using OONI explorer, one can detect successful connections, confirmed instances of censorship, and anomalies when connecting to a specific web domain
from a specific country during a specific time window. For the purposes of this study, I selected data from Taiwan only (sometimes investigating a specific
website in other countries if its global status was unclear), from June 17 to July 17 of 2024, as well as from January 1 to February 1 of 2024 (the month of
the recent Taiwanese national elections). 

No instances of confirmed censorship were detected by OONI in either of the mentioned time periods in Taiwan, so instead, we will focus on rate of anomalies
detected per total number of tests conducted. We will sort the websites into categories of "high anomalies" and "low anomalies," where high anomaly categories 
displayed about 8\% or more of their tests as anomalies. Note that the anomaly occurrence rates between January and June 2024 were nearly identical.

When examining the categories selectable on the OONI Explorer website, the high-anomaly (made up of more than 8\% anomalies) categories were e-commerce,
pornography, provactive attire, alcohol and drugs, file sharing, and gambling. These categories seem to make up a "traditionally seen morally dubious"
supercategory -- pornography, provocative attire, alcohol and drugs, and gambling all seen as vices in most of the world. E-commerce and file sharing are
outliers here; from a censor's point of view, one might want to censor these categories simply to limit user access to unpredictable files or products,
but these could also be pure noise, as anomalies do not indicate a 100\% likelihood of censorship.

Notably absent from the anomalies sighted were categories that might involve political dissent, such as news media, human rights issues, et cetera. This 
absence may indicate that political censorship is not a popular tactic in Taiwan. That being said, there are some specific political categories in which
anomalies seemed to crop up more frequently.

The first of these categories was websites directly operated by the Chinese government, as well as government-sponsored Chinese news websites and (slightly
more surprisingly, although not terribly) the large Chinese Internet marketplace and general platform, Alibaba. This category is relatively unsurprising;
China is Taiwan's greatest geopolitical threat, and Chinese government propaganda would logically be something the Taiwanese government would want to
defend against. Since Chinese media outlets are often seen as mouthpieces for the government due to the strict censorship regime in place there, restricting
the highest-profile Chinese media sites would fit with this philosophy.

The next category was websites affiliated with the North Korean government. This category was also unsurprising; North Korea has an even tighter grip on 
its media than China, and similarly is an ideological and geopolitical enemy of Taiwan. I did not observe any miscellaneous North Korean Internet platforms
incurring anomalies as I did for China, but I am not sure if any such platforms even exist.

Finally, a number of websites related to Islam and Islamic organizations experienced high rates of anomalies, such as the Organization of Islamic Cooperation
and islamdoor.com (as well as a few websites related to general interreligious cooperation). Whether this is an intentional strategy to suppress Islamic
content on the Taiwanese Internet, or purely noise, is unclear.

(A number of Taiwanese news websites also experienced anomalies, but tests related to these websites experienced a high number of \textit{failures} as well,
meaning the data for those websites may simply be garbage.)

I also conducted Tor connectivity tests for these time periods. While the normal Tor test had a low anomaly rate (well within the ranges other countries were
experiencing), the Tor Snowflake test experienced \textit{only} anomalies, which I didn't quite know what to make of. It seems possible that Taiwan is limiting
Tor Snowflake, but not Tor itself, but why is another matter.

Moving on to other websites, Censored Planet had almost no data related to Taiwan, so the remaining data point is a paper published on ICLab entitled
"A Churn for the Better: Localizing Censorship Using Network-level Path Churn and Network Tomography" (\cite{Cho_CoNEXT17}). This paper identified the level
of censorship being performed by autonomous systems across the world using "censorship leakage," a metric by which the larger network was affected by blockages
and slowing in individual autonomous systems. According to this research, Taiwan had a single censoring autonomous system (compared to 6 in China, 0 in the
United States and many other countries, and around 4-5 in a few European countries). This statistic supports the idea that Taiwan is performing some light
censorship on its network, perhaps through an ISP.