\section{Introduction}\label{sec:intro}

When Internet censorship and disinformation are discussed, China is often the country at the forefront, and understandably so. The second-largest
country on Earth by population, with a long history of government control over the Internet domestically as well as a vested interest in influencing
the shape of Internet discourse abroad. 

China's neighbor Taiwan, however, has far less literature on the topic. Taiwan is a vibrant democracy with a crucial role in the geopolitics of both China
(which seeks to annex it) and its rivals. As such, disinformation from both domestic sources and from foreign (especially Chinese) ones are an immediate
concern. Is China seeking to influence Taiwan towards the politics it desires? Are American or other agents acting similarly? How has Taiwan responded to these
challenges, and is government censorship a serious concern? What direction have these forces and trends taken in the past few years?

Some articles have been written about this topic, particularly before and during the 2024 Taiwanese national election. For example, NPR's Emily Feng published
a piece in January of 2024 about the spread of political disinformation on Taiwanese social media platforms, some of it spread by China and others by domestic
media outlets, with the possible goal of affecting election results and undermining the Democratic Progressive Party. Feng describes how the government had 
begun at time of writing to pursue some  anti-disinformation laws and policies in cooperation with Internet companies, but was reluctant to pursue more 
stringent policies that would give the government control over social media during times of emergency. (\cite{npr-taiwan-misinfo})

Likewise, the Associated Press published an article that same month after the election's conclusion, claiming, "In repelling disinformation, Chinese and 
domestic, Taiwan offers an example to other democracies holding elections this year." This article argued that Taiwan's strategy of employing the government,
independent fact-check groups, and private citizens to call out misinformation wherever it arose was highly successful, without needing to employ censorship.
(\cite{ap-taiwan-misinfo})

However, an earlier article from 2019, published in the "Perspectives on Taiwan: Insights from the 2018 Taiwan-U.S. Policy Program" academic journal, claims
that contrary to the jubilant mood expressed in the AP's 2024 article, in 2018, press freedom was under serious threat, with the DPP planning to criminalize
fake news as defined by the government. (\cite{perspectives-taiwan}) With this shift in mind, I feel it is worth examining the trajectory of misinformation
and censorship on the Taiwanese Internet over time -- what caused the government to walk their stance back (or if they did at all!), where they appear to be
heading right now, and if their policy was as effective as it seemed. The results of this project could be used to recommend censorship and fake news policies,
whether in America and other countries (if borrowing from the Taiwanese model), or in Taiwan itself (if the policy was not as successful as it seemed).