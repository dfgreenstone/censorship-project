\section{Conclusion}\label{sec:conclusion}

In conclusion, Taiwan does seem to be performing Internet censorship focused on curtailing narratives from its geopolitical foes, although on a much smaller 
scale than China or other infamous censorship regimes. This claim alone is not terribly surprising; after all, many other countries, including the United 
States, perform some degree of censorship. However, it is a claim worth hearing when the common perception in the media about Taiwan's Internet culture seems
to indicate that censorship is entirely foregone in favor of citizen activism and debunking. While that strategy may be real for the majority of disinformation
spread online in Taiwan, censorship in Taiwan for explicit propaganda from antagonistic governments does seem to exist. In addition, there may be some censorship
of "morally inappropriate" topics as well as unpredictable means of content delivery such as file sharing. That being said, if the media is to be believed,
Taiwan's approach to censorship is working. Citizens do not feel overly censored, and fake news had little impact on the 2024 election. As such, it may be time
to consider Taiwan's strategy of "pre-bunking" and censoring only the most explicit propagandist sources of information -- and why that strategy worked as well
as it did.