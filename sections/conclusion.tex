\section{Conclusion}\label{sec:conclusion}

\subsection{Preparation and Prerequisites}
To complete this project, I'll need to read as many academic and journalistic articles on the topic as possible, and then do my own quantitative research if at
all possible. This will involve looking at specific laws proposed in the Taiwanese assembly, and hopefully exact numbers on Internet users, arrests made,
maybe any throttling or blocking data if I can find it (I'm not terribly confident in that last component). I have done some preliminary research on statistics
and found some numbers on general Internet access and usage rates, but if I can't find any (reputable) censorship statistics, this will be more of a qualitative
synthesis and analysis. My final deliverable in either case would be something akin to a short research paper.

\subsection{Evaluation}

If I can get statistics on Internet censorship or disinformation in Taiwan, I'd do a (proportional) comparison with other countries we have statistics on: likely
the U.S. (which has a similar form of government), and probably any other East Asian countries I could get reliable statistics for (Japan? South Korea?). This
comparison would entail graphing the rates of censorship and disinformation over time, paying special attention to election years and laws or policies 
implemented. In the case where I cannot obtain such data, I would instead harshly interrogate my sources and my sources' sources to ensure I was getting the full
picture, and do ome comparison to the way these issues are discussed when it comes to other countries.

\subsection{Ethics}

My model may suffer from bias from the fact that I am using English-language sources mostly manufactured by Western authors, since I cannot speak Mandarin
(despite living in Taiwan for several years as a child, I did not pick any Mandarin up for very long). This means that there may be incentive by some of these
sources to portray Taiwan in a positive light to bolster pro-Taiwan sentiment for geopolitical reasons, or there may simply be inaccuracies due to ignorance.
That being said, I can't think of any other serious ethical considerations for this project considering the data I'm using is likely to be either official
government data or journalistically sourced, meaning any ethical problems with data collection have essentially already occurred.